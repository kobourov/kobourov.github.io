% This text is proprietary.
% It's a part of presentation made by myself.
% It may not used commercial.
% The noncommercial use such as private and study is free
% May 2007
% Author: Sascha Frank 
% University Freiburg 
% www.informatik.uni-freiburg.de/~frank/
%
% 
\documentclass{beamer}
%\usepackage{beamerthemeshadow}
\usetheme{Frankfurt}
\usepackage{pgf}
\usepackage{graphicx}
\usepackage{verbatim}
\usepackage{eulervm}


\begin{document}
\title{Putting Data on the Map}  
\author{Stephen Kobourov\\University of Arizona}
%\\{\scriptsize\em Joint work with Yifan Hu, Emden Gansner, Chris %Volinsky at AT\&T Research}}
\date{} 
\setbeamertemplate{navigation symbols}{} 
%\setbeamertemplate{footline}[text line]{A Sample Talk}
 \setbeamertemplate{footline}[text line]{ text }

\begin{frame}[plain]
\titlepage\pause
\begin{center}
\vspace{-2cm}
\includegraphics[width=6.5cm]{MapExamples/CatMap}
\end{center}
\end{frame}


%\begin{frame}
%\frametitle{Table of contents}\tableofcontents
%\end{frame} 



\begin{frame}[plain]
\frametitle{Motivation} 
\begin{columns}
\begin{column}{5.3cm}
\begin{itemize}
\item<1-> Given high-dimensional data
\item<2-> Visualize it ``nicely''
\item<3-> What is ``nicely''?
\end{itemize} 
\end{column}
\begin{column}{5.8cm}
\begin{overlayarea}{\textwidth}{6cm} 
\includegraphics<1>[width=5.8cm]{MapExamples/morse_data}
\includegraphics<2>[width=4.5cm]{MapExamples/morse1}
\includegraphics<3>[width=5.0cm]{MapExamples/morse2}
\end{overlayarea}
\end{column}
\end{columns}
\end{frame}


\begin{frame}[plain]\frametitle{Motivation cont.}
\begin{columns}
\begin{column}{5cm}
\begin{itemize}
\item Relational data 
\begin{itemize}
\item represented by graphs
\item objects $\rightarrow$ vertices
\item relationships $\rightarrow$ edges
\end{itemize}

\item Metric data 
\begin{itemize}
\item distance b/n points
\item points $\rightarrow$ vertices
\item distances $\rightarrow$ weighted edges
\end{itemize}
\item Graph drawing
\begin{itemize}
\item can produce great visualization
\item not so good at clusters, neighborhoods
\end{itemize} 
\end{itemize}
\end{column}
\begin{column}{5cm}
\includegraphics[width=4.5cm]{MapExamples/polygons}
\end{column}
\end{columns}
\end{frame}


\begin{frame}[plain]\frametitle{Motivation cont.}
\begin{columns}
\begin{column}{5.5cm}
\begin{itemize}
\item InfoVis rationale
\begin{itemize}
\item <1->humans are good at image processing
\item <2->images can have powerful impact
\item <3->a little help can go a long way
\end{itemize}
\end{itemize}
\end{column}
\begin{column}{5.5cm}
\begin{overlayarea}{\textwidth}{5cm} 
\vspace{-1.5cm}\includegraphics<1->[width=5cm]{MapExamples/captcha2}
\hspace{.2cm}

\includegraphics<2->[width=4cm]{MapExamples/kennedy}
%\includegraphics<2->[width=3.2cm]{MapExamples/RossPerot2}
\hspace{.2cm}

\includegraphics<3->[width=4cm]{MapExamples/TubeMap3}
\end{overlayarea}

\end{column}\end{columns}
\end{frame}



\begin{frame}[plain]\frametitle{Motivation cont.}
\begin{columns}
\begin{column}{5cm}
\begin{itemize}
\item<1-> {\em ``A picture is worth a thousand words''}
\begin{itemize}
\item<2-> how about a graph?
\item<3-> a histogram?
\item<4-> a chart?
\item<5-> a plot?
\end{itemize}
\end{itemize}
\end{column}
\begin{column}{5cm}
\begin{overlayarea}{\textwidth}{6cm} 
\vspace{-.5cm}\includegraphics<2-3>[width=4cm]{MapExamples/wikipedia}

\includegraphics<3-3>[width=4cm]{MapExamples/historyflow}

\includegraphics<4-5>[width=4.5cm]{MapExamples/piechart}

%{<4-5>\em\tiny \copyright graphjam}

\includegraphics<5-5>[width=5cm]{MapExamples/xkcd_decline}

%{<5-5>\em\tiny \copyright xkcd}
\end{overlayarea}

\end{column}\end{columns}
\end{frame}

\begin{frame}[plain]\frametitle{Winning the War}
\includegraphics[width=10cm]{MapExamples/afghanistan_graph}
\end{frame}

\begin{frame}[plain]\frametitle{All Vis, All the Time?}
\includegraphics[width=10cm]{MapExamples/InfoVis}\pause
\hspace{-6cm}\includegraphics[width=3.5cm]{MapExamples/HomerBrain}
\end{frame}

\begin{frame}[plain]\frametitle{Something Different}
\includegraphics[width=10cm]{MapExamples/different}
\end{frame}


\begin{frame}[plain]\frametitle{Less is More}
\includegraphics[width=10cm]{MapExamples/info_confusion}
\end{frame}


\begin{frame}[plain]
\frametitle{Why Maps?}
\begin{columns}
\begin{column}{5cm}
\begin{overlayarea}{\textwidth}{5cm} 
\begin{itemize}
%To paraphrase the soon-to-be junior senator from Minesotta
\item<1-> Good enough
\item<2-> Smart enough
\item<3-> And doggone it, people like them\\
\vspace{1cm}\includegraphics<3->[width=4cm]{MapExamples/dumfuckistan}
\end{itemize}
\end{overlayarea}
%\vspace{2cm} 
\end{column}
\begin{column}{5cm}
\begin{overlayarea}{\textwidth}{5cm} 
\vspace{-1cm}
\includegraphics<1->[width=4.5cm]{MapExamples/electoral}
\newline
\includegraphics<2->[width=4.5cm]{MapExamples/jesusland}
\end{overlayarea}
\end{column}
\end{columns}
\end{frame}


\begin{frame}[plain]
\frametitle{Really, Why Maps?}
\begin{columns}
\begin{column}{4.5cm}
\begin{overlayarea}{\textwidth}{4.5cm} 
\begin{itemize}
%To paraphrase the soon-to-be junior senator from Minesotta
\item<1-> Dimensionality reduction: ${\cal R}^n\rightarrow {\cal R}^2$
\item<2-> Natural extention from graphs 
\item<3-> Explicit clustering: regions and colors
\item<4-> Intuitive and familiar 
%\vspace{1cm}\includegraphics<3->[width=4cm]{MapExamples/dumfuckistan}
\end{itemize}
\end{overlayarea}
%\vspace{2cm} 
\end{column}
\begin{column}{6cm}
\begin{overlayarea}{\textwidth}{6cm} 
\vspace{-1cm}
\includegraphics<1->[width=5cm]{MapExamples/gd_graph}

\includegraphics<2->[width=5cm]{MapExamples/gd_map}
\end{overlayarea}
\end{column}
\end{columns}
\end{frame}


\begin{frame}[plain]\frametitle{Maps, Maps, and More Maps}
\begin{itemize}
\item<1-> Western maps date to 500BC Soleto map
\item<1-> Chinese maps are said to be older
\item<1-> Mercator's map is from 1569
\end{itemize} 
\begin{center}\includegraphics<1>[height=5cm]{MapExamples/Mercator_1569}
\includegraphics<2>[height=5cm]{MapExamples/NatGeo}
\end{center}
\end{frame}


\begin{frame}[plain]\frametitle{Maps, Maps, and More Maps}
\begin{overlayarea}{\textwidth}{6cm} 
\begin{columns}
\begin{column}{5.4cm}
\begin{itemize}
\vspace{-.9cm}\item<1-> Redrawing geographic maps subject to constraints
\item<2-> Cartograms: 2004 US election results
\end{itemize} 
\end{column}
\begin{column}{4.5cm}
\includegraphics<1->[height=2.5cm]{MapExamples/NYTimes2004}
\end{column}
\end{columns}
\begin{center}\includegraphics<2->[height=3.5cm]{MapExamples/NYTimesUSMap}
\includegraphics<2->[height=3.5cm]{MapExamples/NewmanCartogram}
\end{center}
\end{overlayarea}
\end{frame}


\begin{frame}[plain]\frametitle{Maps, Maps, and More Maps}
\begin{itemize}
\item New Yorker '76 vs. Economist '09
\end{itemize} 
\begin{center}\includegraphics[width=5cm]{MapExamples/NewYorkMap}\hspace{.2cm}
\includegraphics[width=5.1cm]{MapExamples/ChinaMap}
\end{center}
\end{frame}

\begin{frame}[plain]\frametitle{Maps, Maps, and More Maps}
\begin{itemize}
\item Whimsical maps: Europe in 1870
% Franco-Prussian war
% Prussia looks like Walrus-mustache Chancellor  Otto Von Bismark
% France is a soldier about to poke Prussia with a sword
% England is an old lady struggling with her Irish dog
\end{itemize} 
\begin{center}\includegraphics[width=9cm]{MapExamples/Europe_map_1870}
\end{center}
\end{frame}


\begin{frame}[plain]\frametitle{Maps, Maps, and More Maps}
\begin{itemize}
\item Whimsical maps: Kalman and Meyerowitz's New Yorkistan
\end{itemize} 
\begin{center}\includegraphics[width=6.5cm]{MapExamples/NewYorkistan}
\end{center}
\end{frame}




\begin{frame}[plain]\frametitle{Maps, Maps, and More Maps}
\begin{itemize}
\item Maps of imagined places: Tolkien's 1930 Middle Earth
\end{itemize} 
\begin{center}
\framezoom<1><2>(2cm,2.4cm)(2.5cm,2cm) 
\framezoom<1><3>(5cm,4.2cm)(2.5cm,2cm) 
\pgfimage[height=7cm]{MapExamples/MapMiddleEarth}
\end{center}
\end{frame}

\begin{frame}[plain]\frametitle{Maps, Maps, and More Maps}
\begin{itemize}
\item Maps of imagined places: Woelfle's 1938 map of books
% Makulaturia: wastepaper land
% Poesia: poetry
% Verbotten Provanz: forbidden province
% Lesser Republic: reader's republic
% Recenscentia: reviewer's realm
\end{itemize} 
\begin{center}\includegraphics[height=7cm]{MapExamples/bucherlandes}
\end{center}
\end{frame}


\begin{frame}[plain]
\frametitle{Maps, Maps, and More Maps}
\begin{itemize}
\item Maps of imagined places: online communities circa 2007 
\end{itemize}
\begin{center}
\framezoom<1><2>(1.4cm,1.2cm)(5cm,4cm) 
\framezoom<1><3>(2.55cm,3.9cm)(5cm,4cm) 
%\framezoom<1><4>(3cm,2cm)(3cm,2cm) 
\pgfimage[height=7cm]{MapExamples/xkcd}{\em\tiny \copyright xkcd}
\end{center}
\end{frame} 


\begin{frame}[plain]\frametitle{NetFlix Contest}
\begin{columns}
\begin{column}{6.5cm}
\begin{itemize}
\item <1-> Data Set
\begin{itemize}
\item 20,000 movies
\item 500,000 users
\item 100,000,000 ratings
\end{itemize}
\item <2-> Challenge
\begin{itemize}
\item improve recommender's {\em RMSE}
\item CineMatch: 10\% over trivial
\item goal: do another 10\% better
\item over 2,000 teams participated
\end{itemize}
\item <3-> History
\begin{itemize}
\item data released 10/02/06
\item CineMax beaten on 10/08/06
\item BellKor: 8.4\% on 10/08/07
\item BellKor+: 9.4\% on 10/08/08
\item BellKor++: 10.1\% on 06/26/09
\end{itemize}
\item <4->
\end{itemize}
\end{column}
\begin{column}{5.5cm}

\begin{overlayarea}{\textwidth}{5cm} 
\vspace{-1cm}\includegraphics<1->[width=3.5cm]{MapExamples/netflix-logo}
\hspace{.2cm}

%\includegraphics<2->[width=3.2cm]{MapExamples/netflix-logo}
\vspace{1cm}

\includegraphics<3-3>[height=3cm]{MapExamples/bob}
\includegraphics<3-3>[height=3cm]{MapExamples/yehuda}
\includegraphics<4->[height=3cm]{MapExamples/posums}
\end{overlayarea}
\end{column}
\end{columns}
\end{frame}



\begin{frame}[plain]\frametitle{Netflix $\rightarrow$ UVerse}
\begin{columns}
\begin{column}{\textwidth}
\begin{itemize}
 \item MyVerse project
 \begin{itemize}
  \item UVerse TV recommenders
  \item collaborative filtering {\em\tiny (SVD, kNN)}
  \item content based {\em\tiny (Bayesian)} 
 \end{itemize}
 \item Data in 2008
 \begin{itemize}
  \item 755,000 set-top boxes
  \item 37,000 unique shows
  \item 161,000,000 monthly events
  \item 10 hours of data per day
 \end{itemize}
 \item Metric data
 \begin{itemize}
  \item TV shows $\rightarrow$ points
  \item similarity b/n shows $\rightarrow$ distances
 \end{itemize}
\end{itemize}
\end{column}
\begin{column}{2cm}
\hspace{-4cm}\includegraphics[width=5cm]{MapExamples/myverse}
\end{column}
\end{columns}
\end{frame}


\begin{frame}[plain]\frametitle{Local Motivation}
\begin{itemize}
\item U-Verse data
\end{itemize} \begin{center}
\framezoom<1><2>(6.7cm,3.5cm)(1.5cm,1cm) 
\framezoom<1><3>(.8cm,4.4cm)(2cm,1cm) 
\pgfimage[height=5cm]{MapExamples/uverse1k}
\end{center}
\end{frame}

\begin{frame}[plain]\frametitle{Local Motivation}

\begin{itemize}
\item Why is a show recommended?

\item Explanation for recommending {\small \em Real Stories of the Highway Patrol}

\includegraphics[width=8cm]{MapExamples/table}\pause

\includegraphics[width=7cm]{MapExamples/correlation1} \copyright xkcd


%\item But once again, why?
%\item How are these shows related?
%\item Even the experts are not always sure
%\\ (or what's the matter with {\small \em Naked and Nasty}? )
\end{itemize}
\end{frame}









\begin{frame}[plain]\frametitle{GMap}
\begin{columns}
\begin{column}{8cm}
\begin{itemize}
 \item<1-> Why GMap?
 \begin{itemize}
  \item YAGDT
  \item Data Map
  \item {\bf G}raph {\bf Map}
  \item {\bf G}eo{\bf G}raphic {\bf Map}
 \end{itemize}
\end{itemize}
\end{column}
\begin{column}{4cm}
\hspace{-4cm}\includegraphics[width=7cm]{MapExamples/TVLand}
%[width=5cm]{MapExamples/gd_8comp}
\end{column}
\end{columns}

\begin{columns}
\begin{column}{\textwidth}
\begin{itemize}
\item <2->
What is GMap? embedding + clustering + mapping 
 \begin{itemize}
  \item different algorithms: embedding, clustering, mapping
  \item different ordering: embed $\rightarrow$ cluster, or cluster $\rightarrow$ embed
  \item different mapping: continents, islands, borders, colors
  \item different interpretation: font sizes, 3D terrain, colors
 \end{itemize}
\end{itemize}
\end{column}
\begin{column}{1cm}
\end{column}
\end{columns}
\end{frame}






\begin{frame}[plain]\frametitle{TVLand}
\begin{center}
\framezoom<1><2>(4.2cm,1.2cm)(2cm,1cm) %ToddlerSprawl
\framezoom<1><3>(.7cm,4cm)(2cm,1cm) %Premium Peninsula
\framezoom<1><4>(4.8cm,5.7cm)(2cm,1cm) %Renovation
\framezoom<1><5>(6.3cm,5.7cm)(2cm,1cm) %Hung'ry
\framezoom<1><6>(7.3cm,3.6cm)(2cm,1cm) %SoapLand
\framezoom<1><7>(2.4cm,4.6cm)(2cm,1cm) %Leftbank
\framezoom<1><8>(8.2cm,4.3cm)(2cm,1cm) %RightBank
\framezoom<1><9>(7.9cm,1.1cm)(2cm,1cm) %HighLand
\framezoom<1><10>(4.5cm,2.8cm)(2cm,1cm) %MidLands
\framezoom<1><11>(5.4cm,3.7cm)(2cm,1cm) %LowLands
\pgfimage[height=7cm]{MapExamples/TVLand}
\end{center}
\end{frame}

\begin{frame}[plain]\frametitle{TVLand Compass}
\begin{center}
\includegraphics<1->[height=7cm]{MapExamples/TVLand}
%\pgfdeclaremask{mymask}{compass.mask}
%\pgfimage[mask=mymask,interpolate=true]{compass}
\llap{\includegraphics<2->[height=2cm]{MapExamples/compass2}}
% age: who else but grumpy old men watch O'Reilly and the Fox News at 5am
% fashion and entertainment --> sports and news
\end{center}
\end{frame}

\begin{frame}[plain]\frametitle{TVLand Compass}
\begin{center}
\includegraphics<1->[height=7cm]{MapExamples/uverse_rec_heat_map}
\end{center}
\end{frame}

\begin{comment}

\begin{frame}[plain]\frametitle{BookLand}
\begin{center}
\framezoom<1><2>(5cm,3.5cm)(2cm,1cm) %Americana
\framezoom<1><3>(3.1cm,3.cm)(2cm,1cm) %Shakespearea
\framezoom<1><4>(3.4cm,4.4cm)(2cm,1cm) %Victoriana
\framezoom<1><5>(5.1cm,5.2cm)(2cm,1cm) %Thespia and Coelholand
\framezoom<1><6>(2cm,1.9cm)(2cm,1cm) %Graecoromania
\framezoom<1><7>(5.4cm,1.8cm)(2cm,1cm) %Russiana
\framezoom<1><8>(5.8cm,.8cm)(2cm,1cm) %Fringistan
%\framezoom<1><9>(3.4cm,.4cm)(2cm,1cm) %Feministan
\pgfimage[height=6cm]{MapExamples/BookLand}
\end{center}
\end{frame}

\end{comment}

\begin{frame}[plain]\frametitle{When you have a hammer...}
\begin{center}
\includegraphics[height=6cm]{MapExamples/BookLand}
\end{center}
\end{frame}


\begin{frame}[plain]\frametitle{When you have a hammer...}
\begin{center}
\includegraphics[height=7cm]{FIGURES/LastFmTop2000}
\end{center}
\end{frame}

\begin{frame}[plain]\frametitle{When you have a hammer...}
\begin{center}
\includegraphics[height=7cm]{FIGURES/UniversityMap}
\end{center}
\end{frame}

\begin{frame}[plain]\frametitle{When you have a hammer...}
\begin{center}
\includegraphics[height=7cm]{FIGURES/NetflixTop1000}
\end{center}
\end{frame}


\begin{frame}[plain]\frametitle{Under the Hood}

\begin{overlayarea}{\textwidth}{5cm} 
\begin{center}
\vspace{-1.3cm}
\includegraphics<1->[width=6cm]{MapExamples/car1}
\vspace{.7cm}
\includegraphics<2->[width=6cm]{MapExamples/car2}
\end{center}
\end{overlayarea}
\end{frame}



\begin{frame}[plain]\frametitle{GMap Overview}

\begin{columns}
\begin{column}{7.3cm}
\begin{itemize}
\item<1-> Embedding
\begin{itemize}
\item PCA, MDS, LLE, IsoMap, $\dots$
\item GMap: {\em scalable force-directed method}
\end{itemize}
\item<2-> Clustering
\begin{itemize}
\item agglomerative, $k$-means, spectral, $\dots$
\item GMap: {\em modularity clustering}
\end{itemize}
\item<3-> Mapping 
\begin{itemize}
\item no previous work on graph$\rightarrow$ map
\item GMap: {\em modified Voronoi Diagram}
\end{itemize}
\end{itemize}
\end{column}
\begin{column}{4cm}
\begin{overlayarea}{\textwidth}{5cm} 
\begin{center}
\vspace{-1.3cm}
\includegraphics<1->[width=3.5cm]{YIFAN/DATA/karate_graph}
\vspace{.2cm}
\includegraphics<2->[width=3.5cm]{YIFAN/DATA/karate_graph_cluster}
\vspace{0cm}
\includegraphics<3->[width=5cm]{YIFAN/DATA/karate_map2}

\end{center}
\end{overlayarea}
\end{column}
\end{columns}
\end{frame}









\begin{frame}[plain]\frametitle{GMap: Embedding}

\begin{itemize}
\item Given a graph $G=(V,E)$
\begin{itemize}
\item places vertices as points in ${\cal R}^d$
\item route edges in ${\cal R}^d$
\end{itemize}
\item Force directed methods define an energy function on layouts
\begin{itemize}
\item based on attractive/repulsive forces (Fruchterman-Reingold)
\item based on graph distances (Kamada-Kawaii)
\end{itemize}
\item Energy model
\begin{itemize}
\item iterative improvement
\item minimal energy $\Rightarrow$ good layout
\end{itemize}
\end{itemize}
\vspace{.4cm}
\begin{center}\includegraphics[width=2.1cm]{MapExamples/spring1}\includegraphics[width=2.1cm]{MapExamples/spring2}\includegraphics[width=2.1cm]{MapExamples/spring3}\includegraphics[width=2.3cm]{MapExamples/spring4}\includegraphics[width=2.1cm]{MapExamples/spring5}\end{center}
\end{frame}


\begin{comment}
\begin{frame}[plain]\frametitle{GMap: Embedding, cont.}
\vspace{.1cm}
\begin{block}{Demo}{Force-directed algorithms in action!}\end{block}

\vspace{1cm}

\begin{center}\includegraphics[width=2.1cm]{MapExamples/spring1}\includegraphics[width=2.1cm]{MapExamples/spring2}\includegraphics[width=2.1cm]{MapExamples/spring3}\includegraphics[width=2.3cm]{MapExamples/spring4}\includegraphics[width=2.1cm]{MapExamples/spring5}\end{center}
\end{frame}
\end{comment}


\begin{frame}[plain]\frametitle{... and now for some math...}

\begin{center}
\includegraphics[width=6cm]{MapExamples/Find_X}
\end{center}
\end{frame}

\begin{frame}[plain]\frametitle{GMap: Embedding, cont.}

\begin{itemize}
\item Fruchterman-Reingold: balance of attraction and repulsion

\begin{block}{}
%\begin{eqnarray*}
$$F(v)=F_r(v)+F_a(v)$$
$$F_r(v)=\sum_{\forall u\in V}\frac{\kappa^2}{\|pos[u]-pos[v]\|^2}(pos[u]-pos[v])$$
$$F_a(v)=\sum_{u\in Adj(v)}\frac{\|pos[u]-pos[v]\|^2}{\kappa^2}(pos[u]-pos[v])$$
%\end{eqnarray*}
\end{block}

\item $\kappa=\sqrt{A_{frame}/|E|}$, ideal edge length

\item Kamada-Kawai: match Euclidean distance to graph distance
\end{itemize}
\begin{block}{}$$ F(v)=\sum_{u\in V}\left( \frac{\|pos[u]-pos[v]\|^2}{(\kappa\times dist_G(u,v))^2} -1 \right)(pos[u]-pos[v])$$\end{block}

\end{frame}






\begin{frame}[plain]
\frametitle{GMap: Clustering}
\begin{itemize}
\item Modularity
\begin{itemize}
\item measure of the quality of vertex grouping
\item high edge density {\em within} groups
\item 
low edge density {\em between} groups
\end{itemize}
\begin{block}{}
$$\frac{1}{2|E|} \sum_{\substack{\forall(u,v)}} \left\{ e(u,v) - \frac{deg(u)deg(v)}{2|E|} \right\} \delta(c_u, c_v)$$
\end{block}
\item Computing modularity
\begin{itemize}
\item value: $[-1, 1]$
\item opt. modularity is NP Hard
\item spectral heuristics are fast
\end{itemize}
\end{itemize}
\vspace{-1.5cm}\hspace{5.6cm}\includegraphics[width=6cm]{YIFAN/DATA/karate_graph_cluster}

\end{frame}


\begin{frame}[plain]
\frametitle{GMap: Mapping}

\begin{columns}
\begin{column}{7cm}
\begin{center}\includegraphics[width=5cm]{YIFAN/DATA/karate_graph_cluster}
\end{center}
\begin{itemize}
\item{} Use Voronoi Diagram for borders
\item{} Add bounding box for a finite VD
\end{itemize}
\vfill
\end{column}
\pause
\begin{column}{5cm}
\begin{center}
\includegraphics[width=4cm]{YIFAN/DATA/karate_map_bad.png}
\end{center}
\end{column}
\end{columns}
\end{frame}

\begin{frame}[plain]
\frametitle{GMap: Mapping, cont.}
\begin{itemize}
\item Problems with simple VD
\begin{itemize}
\item{} outer boundary is not ``form fitting''
\item{} inner boundaries are too ``mid-western''
\item{} not enough space for important nodes
\end{itemize}
\end{itemize}
\begin{center}
\includegraphics[width=5cm]{YIFAN/DATA/example_naive.png}
\end{center}

\end{frame}


\begin{frame}[plain]
\frametitle{GMap: Mapping, cont.}
\begin{itemize}
\item Our solution
\begin{itemize}
\item{} add ``box'' points around each label's bounding box
\item{} add more ``noise'' points everywhere\\ (some distance away from real \& box points)
\item{} generate Voronoi Diagram
\item<2-> merge cell of the same color to form countries
\end{itemize}
\end{itemize}
\begin{columns}
\begin{column}{5cm}
\begin{center}
\includegraphics<1->[width=5cm]{YIFAN/DATA/example_voro.png}
\end{center}
\end{column}
\begin{column}{4cm}
\pause
\begin{center}
\includegraphics<2->[width=4cm]{YIFAN/DATA/example_final.png}
\end{center}
\end{column}
\end{columns}
\end{frame}

\begin{frame}[plain]
\frametitle{GMap: Mapping, cont.}
\begin{columns}
\begin{column}{6cm}
\begin{itemize}
\vspace{-1cm}
\item<1->Real and dummy points
\begin{itemize}
\item real: real labeled data points
\item box: proportional regions
\item noise: form-fitting coasts
\end{itemize}
\vspace{1cm}
\item<2->Merge adjacent cells 
\begin{itemize}
\item European-style borders
\item Natural-looking coastlines
\item Some fragmentation possible
\end{itemize}
\end{itemize}
\end{column}
\begin{column}{8cm}
\begin{overlayarea}{\textwidth}{8cm}
\vspace{-.7cm}
\begin{center}
\includegraphics<1->[width=7.5cm]{YIFAN/DATA/karate_map_voro.png}
\end{center}
\vspace{-1.cm}
\begin{center}
\includegraphics<2->[width=7.5cm]{YIFAN/DATA/karate_map.png}
\end{center}
\end{overlayarea}
\end{column}
\end{columns}
\end{frame}

\begin{frame}[plain]
\frametitle{GMap: Coloring}
\begin{itemize}
\item{} Four Color Theorem: any map can be colored with {\bf 4} colors
\begin{itemize}
\item{} implicit assumption that countries are contiguous
\end{itemize}
\item{} Our ``countries'' may not be contiguous
\begin{itemize}
\item{} need as many colors as the number of clusters%, $k$
\item careful with colors for adjacent countries
\end{itemize}

\item{} Color selection
\begin{itemize}
\item ``map-like'' pastel colors
\item  5 color paletter from ColorBrewer

\begin{center}
  \vspace{-.5cm}\includegraphics[width=10cm,height=.5in]{YIFAN/DATA/coloring_bar.png}\vspace{-.5cm}
\end{center}
\item{} blend them to get as many colors as needed

\begin{center}
  \vspace{-.5cm}\includegraphics[width=10cm,height=.5in]{YIFAN/DATA/coloring_bar2.png}\vspace{-.5cm}
\end{center}

\item{} discrete color space; any two consecutive colors are very similar
\end{itemize}
\end{itemize}

\end{frame}

\begin{frame}[plain]
\frametitle{ GMap: Coloring, cont.}
\begin{itemize}

\item Color assignment to countries
\begin{itemize}
\item create country graph $G_c=(V_c,E_c)$
\item vertices in $V_c$ are countries
\item edge $(i,j)\in E_c \iff$ countries $i$ and $j$ share a border
\item vertex labeling $c:V\rightarrow \{1,2, \dots, |V|\}$ should maximize difference b/n adjacent countries
\end{itemize}

\begin{block}{}
$\max  \min_{(i,j)\in E_c} w_{i,j} \left|c_i - c_j\right|,\quad\left\{c_1, c_2, \ldots, c_{k}\right\}\ \text{is\ a\ permutation}$
\end{block}


\end{itemize}

\begin{center}
\begin{table}
 \begin{tabular}{cc}
 \includegraphics[width=5cm]{YIFAN/DATA/coloring_random.png} &  \includegraphics[width=5cm]{YIFAN/DATA/coloring_optimal.png}\\
 random & after optimization\\
\end{tabular}
\end{table}
\end{center}

\end{frame}

\begin{frame}[plain]
\frametitle{ GMap: Coloring, cont.}
\begin{itemize}

\item{} The complementary problem:
\begin{itemize}
\item minimum bandwidth ordering for sparse matrices
\item NP hard
\end{itemize}
\begin{block}{}$$\min  \max_{(i,j)\in E}w_{i,j} \left|c_i - c_j\right|\quad \longleftrightarrow \quad \max \min _{(i,j)\in E_c}w_{i,j} \left|c_i - c_j\right|$$ \end{block}

\item{} We use a combination of two heuristics:
\begin{itemize}
\item spectral method
\item greedy improvement
\end{itemize}

%$$\max  \min _{(i,j)\in E}w_{i,j} \left|c_i - c_j\right|$$
%as

\begin{block}{}
$$\max  \sum_{(i,j)\in E_c} w_{i,j} \left(c_i-c_j\right)^2$$
{subject to} $$\sum_{k\in V}  c_k = 1$$
\end{block}

%\item{} The solution: the largest eigenvector of the weighted %Laplacian.


\end{itemize}
\end{frame}

\begin{frame}[plain]
\frametitle{GMap: Coloring, random}
\begin{center}
\includegraphics[width=11cm]{YIFAN/DATA/uverse_1000_country_labels_70in.png}
\end{center}
\end{frame}

\begin{frame}[plain]
\frametitle{GMap: Coloring, heuristic}
\begin{center}
\includegraphics[width=11cm]{YIFAN/DATA/uverse_1000_country_labels_70in_coloring.png}
\end{center}
\end{frame}



\begin{frame}[plain]
\frametitle{Planar Maps}

\begin{itemize}
\item contact graphs
\item vertices are represented by geometrical objects
\item edges correspond to two objects touching
\end{itemize}

\begin{center}
    \includegraphics[scale=.5]{FIGURES/G1.pdf}
    \includegraphics[scale=.5]{FIGURES/G1Maps.pdf}
    \includegraphics[scale=.5]{FIGURES/G1Triangles.pdf}
 \end{center}
\end{frame}

\begin{frame}[plain]
\frametitle{Planar Maps cont.}
\begin{itemize}
\item Vertices: polygons
\item Edges: non-trivial borders
\item Complexity and convexity
\end{itemize}

\begin{center}
    \includegraphics[scale=.5]{FIGURES/G1.pdf}
    \includegraphics[scale=.5]{FIGURES/G1Maps.pdf}
    \includegraphics[scale=.5]{FIGURES/G1Triangles.pdf}
 \end{center}
\end{frame}

\begin{frame}[plain]
\frametitle{Hexagonal map algorithm}
\begin{itemize}
\item If regions are convex, six sides are needed
\item Six sides also suffice for every planar graph
\item Constructive algorithm:
\end{itemize}
\begin{center}
\includegraphics{FIGURES/constructionExampleA}
\includegraphics{FIGURES/constructionExampleB}
\includegraphics{FIGURES/constructionExampleC}
\includegraphics{FIGURES/constructionExampleD}
 \end{center}
\end{frame}



\begin{frame}[plain]
\frametitle{Examples...}
\begin{center}
\includegraphics[scale=.3]{FIGURES/example2.pdf}
\includegraphics[width=.4\textwidth]{FIGURES/tqg_3.pdf}
 \end{center}
\end{frame}

\begin{frame}[plain]
\frametitle{Examples...}
\begin{center}
\includegraphics[scale=.3]{FIGURES/example3.pdf}
\includegraphics[width=.4\textwidth]{FIGURES/tqg_2.pdf}
 \end{center}
\end{frame}

\begin{frame}[plain]
\frametitle{Lower Bound}
Consider this graph:
\begin{center}
\includegraphics[scale=.3]{FIGURES/example1.pdf}
\includegraphics[width=.4\textwidth]{FIGURES/tqg_1.pdf}
 \end{center}
\begin{itemize}
\item Add a vertex in each internal face and triangulate
\item Argue 6 sides are needed...
\end{itemize}

\end{frame}

\begin{frame}[plain]\frametitle{Future Work}
\begin{columns}
\begin{column}{8cm}
\vspace{-1cm}
\begin{itemize}
\item Experiments
\begin{itemize}
\item what tasks are suited to maps?
\item how much better are maps?
\item what types of maps are better?
\end{itemize}
\item Interactive Map Interface
\begin{itemize}
\item semantic zooming
\item control edge clutter, fragmentation
\item map overlays: heatmap, relief
\end{itemize}
\item Theoretical Questions:
\begin{itemize}
\item touching polygonal representations
\item characterizations
\item recognition algorithms
\item coloring problem
\end{itemize}
\end{itemize}
\end{column}
\begin{column}{8cm}
\begin{overlayarea}{\textwidth}{8cm}
\vspace{1cm}

\includegraphics<2->[width=4cm]{FIGURES/spectrum}

\end{overlayarea}
\end{column}
\end{columns}
\hspace{1cm}\url{www.cs.arizona.edu/~kobourov/PROJECTS/maps.html}

\end{frame}

\begin{frame}[plain]\frametitle{Thanks!}
\vspace{.1cm}
\begin{block}{D. Knuth}{\em Graph drawing is the best possible field I can think of. It merges aesthetics, mathematical beauty and wonderful algorithms. It therefore provides a harmonic balance between the left and right brain parts.}\end{block}

\begin{center}
\includegraphics[width=3cm]{MapExamples/sierpinski4}
\end{center}
\end{frame}


\end{document}


\begin{comment}

\begin{frame}[plain]\frametitle{Fragmentation}
\begin{center}
\includegraphics<1-1>[width=11cm]{YIFAN/DATA/gd_8comp2}
\end{center}
\end{frame}


\begin{frame}[plain]\frametitle{Fragmentation, cont.}
\begin{itemize}
\item Why are some countries fragmented?
\begin{itemize}
\item one good reason: connectivity dictates it
\item some not-so-good reasons:
\begin{itemize}
\item poor match between embedding and clustering
\item poor dimensionality reduction
\item label overlap removal
\end{itemize}
\end{itemize}
\end{itemize}
\begin{center}
\includegraphics<1-1>[width=9cm]{YIFAN/DATA/north}
\end{center}
\end{frame}


\begin{frame}[plain]\frametitle{Fragmentation, cont.}
\begin{center}
\includegraphics<1-1>[width=11cm]{YIFAN/DATA/gd_8comp2}
\vspace{0.cm}
\includegraphics<2-2>[width=11cm]{YIFAN/DATA/gd_8comp_neato2}
\end{center}
\end{frame}

\end{comment}

